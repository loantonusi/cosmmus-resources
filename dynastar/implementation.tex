%!TEX root =  main.tex
\section{Implementation}
\label{sec:implementation}


Our  \dynastar prototype is written as a
Java 8 library.
All code, including the experiment harness,
is publicly available with an open-source
license.\footnote{https://github.com/anonymous}
Application designers who use \dynastar
 to implement a replicated service must extend three key classes:
 \begin{itemize}
 \item[--] \emph{PRObject}: provides a common interface for (partially) replicated data items.
 \item[--] \emph{PartitionStateMachine}: encapsulates the logic of the server
   proxy. Note that the server logic is written without knowledge of the actual partitioning scheme. The \dynastar library
   handles all communication between partitions and the oracle transparently.
 \item[--] \emph{OracleStateMachine}: computes the mapping of objects to partitions.
Our default implementation uses METIS\footnote{http://glaros.dtc.umn.edu/gkhome/views/metis} to provide an optimal partitioning based on the workload graph.
 \end{itemize}

 We note one important implementation detail.  The oracle is
 multi-threaded, and can service requests while computing a new
 partitioning concurrently. To ensure that all replicas start using
 the new partitioning consistently, the oracle identifies each
 partitioning with a unique id.  When an oracle replica finishes a
 repartitioning, it atomically multicasts the id of the new
 partitioning to all replicas of the oracle.  The first delivered id
 message defines the order of the new partitioning with respect to
 other oracle operations.

%% In this section, we describe \libname{}, a library that implements
%% both \ssmr{}, \dssmr{} and \dynastar{}.  We also present \appname{}, a
%% scalable social network application built with
%% \libname{}. \libname\ and \appname\ were both implemented in Java.

%% \subsection{\libname}

%% \libname{} implements functionalities of static and dynamic mapping,
%% and supports centralized and decentralized partitioning schemes.
%% There are three classes that the developer (i.e., application
%% designer) must extend to implement a replicated service with
%% \libname{}: PRObject, PartitionStateMachine, OracleStateMachine.

%% The \emph{PRObject class} represents any kind of data item that is
%% part of the service state. The state is partially replicated (i.e.,
%% objects are distributed among partitions). Therefore, when executing a
%% command, a replica might not have local access to some of the objects
%% involved in the execution of the command. The developer informs
%% \libname{} which object classes are partially replicated by extending
%% the PRObject class. Each object of such a class is stored locally or
%% remotely, but the application code is agnostic to the location of an
%% object. All calls to methods of such objects are intercepted by
%% \libname{}, transparently to the developer.

%% The \emph{PartitionStateMachine class} represents logic of the server
%% proxy. The application server class must extend the
%% PartitionStateMachine class. To execute commands, the developer must
%% provide an implementation for the method executeCommand(Command). The
%% code for such a method is agnostic to the existence of partitions. In
%% other words, developer programs for classical state machine
%% replication (i.e., full replication). \libname{} is responsible for
%% handling all communication between partitions and oracle
%% transparently. Objects that are involved in application's command will
%% be available to the partition at the time it is accessed by the
%% partitions. To start the server, method runStateMachine() is
%% called. Method createObject() also needs to be implemented, where the
%% developer defines how new state objects are loaded or created.


%% The \emph{OracleStateMachine class} defines the oracle, which is
%% deployed as a partition.  Because the oracle is in charge of moving
%% the PRObjects to the appropriate partitions, it requires a partitioner
%% that will provide for each object a desired location. A default
%% partitioner is implemented using METIS, a state-of-the-art graph
%% partitioner, but any algorithm that takes as input a graph and outputs
%% a mapping of objects to partitions is a valid partitioner.  The
%% workload graph is built by aggregating hints about the objects' access
%% patterns, and the mapping it returns indicates the ideal partition for
%% each object. Since the partitioner is executed in every oracle
%% replica, its output should be deterministic so all the replicas
%% transition to the same state after the partitioning.  The developer
%% can override the default algorithm to partition the graph and the
%% oracle is agnostic to its internal behavior.  At any time, the
%% application can request a repartitioning of the graph.


\subsection{\appname\ social network service}
\label{sec:imp:\appname}

For the purposes of our evaluation, we have developed a Twitter-like
social network service using the \dynastar{} library.  In our social
network, users can follow, unfollow, post, or read other users' tweets
according to whom the user is following. Like Twitter, users are
constrained to posting 140-character messages.

Each user in the social network corresponds to a node in a graph. If
one user follows another, then a directed edge is created from the
follower to the followee. Each user has an associated \emph{timeline},
which is a sequence of post messages from the people that the user
follows. Thus, when a user issues a post command, it results in
writing the message to the timeline of all the user's followers.  In
contrast, when a user wants to read their own timeline, they only need
to access the state associated with their own node.

Overall, in \appname, post, follow or unfollow commands can lead to
object moves.  Follow and unfollow commands can involve at most two
partitions, while posts may require object moves from many partitions.

Of course, other implementations are possible. But, given that users
in social networks spend most of the time reading (i.e., performing
getTimeline)~\cite{facebookTAO}, it is useful to implement getTimeline
as an efficient single-partition command.


\subsection{Alternative Systems}

Throughout our evaluation, we compare \dynastar{} to two alternative
systems: S-SMR and DS-SMR. For performing our experiments, we used the
implementations of the two systems that are available in public
repositories.\footnote{https://bitbucket.org/usi-dslab/ds-smr} Since
all three systems (S-SMR, DS-SMR, and \dynastar) provide a common
application-level interface, there was no additional work required to
port the social network application.

