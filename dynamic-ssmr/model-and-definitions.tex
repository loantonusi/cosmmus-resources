\section{System model and definitions}
\label{sec:sysmodel}

We consider a distributed system consisting of an unbounded set of client processes $\ccm = \{c_1, c_2, ...\}$ and a bounded set of server processes (replicas) $\ssm = \{s_1, ..., s_n\}$. 
Set $\ssm$ is divided into disjoint groups of servers $\ssm_0, ..., \ssm_k$.
Processes are either \emph{correct}, if they never fail, or \emph{faulty}, otherwise. 
In either case, processes do not experience arbitrary behavior (i.e., no Byzantine failures).

Processes communicate by message passing, using either one-to-one or one-to-many communication.
The system is asynchronous: there is no bound on message delay or on relative process speed.
One-to-one communication uses primitives $send(p,m)$ and $receive(m)$, where $m$ is a message and $p$ is the process $m$ is addressed to. 
If sender and receiver are correct, then every message sent is eventually received. 
%
One-to-many communication relies on reliable multicast and atomic multicast,\footnote{Solving atomic multicast requires additional assumptions~\cite{CT96,FLP85}. In the following, we simply assume the existence of an atomic multicast oracle.}
defined in sections~\ref{sec:rmcast} and \ref{sec:amcast}, respectively.
%Atomic broadcast is a special case of atomic multicast in which there is a single group with all servers.

Our consistency criterion is linearizability.
A system is \emph{linearizable} if there is a way to reorder the client commands in a sequence that (i)~respects the semantics of the commands, as defined in their sequential specifications, and (ii)~respects the real-time precedence of commands~\cite{Attiya04}.

\subsection{Reliable multicast}
\label{sec:rmcast}

To reliably multicast a message $m$ to a set of groups $\gamma$, processes use primitive \rmcast$(\gamma, m)$.
Message $m$ is delivered at the destinations with \rmdel$(m)$.
Reliable multicast has the following properties:

\begin{itemize}

    \item[--] If a correct process \rmcast{}s $m$, then every correct process in $\gamma$ \rmdel{}s $m$ \emph{(validity)}.
    
    \item[--] If a correct process \rmdel{}s $m$, then every correct process in $\gamma$ \rmdel{}s $m$ \emph{(agreement)}.
    
    \item[--] For any message $m$, every process $p$ in $\gamma$ \rmdel{}s $m$ at most once, and only if some process has \rmcast{} $m$  to $\gamma$ previously \emph{(integrity)}.
    
\end{itemize}

\subsection{Atomic multicast}
\label{sec:amcast}

To atomically multicast a message $m$ to a set of groups $\gamma$, processes use primitive \amcast$(\gamma, m)$.
Message $m$ is delivered at the destinations with \amdel$(m)$.
Atomic multicast ensures the following properties:

\begin{itemize}
    
    \item[--] If a correct process \amcast{}s $m$, then every correct process in $\gamma$ \amdel{}s $m$ \emph{(validity)}.
    
    \item[--] If a process \amdel{}s $m$, then every correct process in $\gamma$ \amdel{}s $m$ \emph{(uniform agreement)}.
    
    \item[--] For any message $m$, every process $p$ in $\gamma$ \amdel{}s $m$ at most once, and only if some process has \amcast{} $m$ to $\gamma$ previously \emph{(integrity)}.
    
    \item[--] No two processes $p$ and $q$ in both $\gamma$ and $\gamma'$ \amdel\ $m$ and $m'$ in different orders; also, the delivery order is acyclic. \emph{(atomic order)}.
    
\end{itemize}

Atomic broadcast is a special case of atomic multicast in which there is a single group of processes.